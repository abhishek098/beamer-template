\documentclass{beamer}


\usepackage[utf8]{inputenc}
\usepackage{amsmath}
\usepackage{amsfonts}
\usepackage{amssymb}
\usepackage{graphicx}
\usepackage{ragged2e}  % `\justifying` text
\usepackage{booktabs}  % Tables
\usepackage{tabularx}
\usepackage{tikz}      % Diagrams
\usetikzlibrary{calc, shapes, backgrounds}
\usepackage{amsmath}
\usepackage{amssymb}
\usepackage{dsfont}
\usepackage{url}       % `\url
\usepackage{listings}  % Code listings
\usepackage[T1]{fontenc}


\usepackage{theme/beamerthemehbrs}

\author[]{Abhishek Padalkar}

\title{Dynamic Motion Primitives}
\subtitle{Research and Development Project}
\institute[HBRS]{Hochschule Bonn-Rhein-Sieg}
\date{\today}
\subject{Test beamer}

% \thirdpartylogo{path/to/your/image}


\begin{document}
	{
	\begin{frame}
	\titlepage
	\end{frame}
	}
	
	\begin{frame}{Motivation}
		\begin{itemize}
			\item Humans \textbf{learn} variety of motion and use them in similar situations. 
			\item Biological motions consist of motion primitives. 
			\item Concept of motion primitives can be adopted for robots. 
			\item Learned motion primitives can be combined to do complex task. 
			\item 
		\end{itemize}
	\end{frame}
	
	
	\begin{frame}{Formulation of DMP}
		\begin{equation}\label{DMP_1}
		\tau\dot{z} = \alpha_{z}(\beta_{z}(g - y) - z) + f(x)
		\end{equation}
		\begin{equation}\label{DMP_2}
		\tau \dot{y} = z
		\end{equation}
		\begin{equation}\label{forcing_term}
		f(x) = \frac{\sum_{i=1}^{N}\psi_{i}(x)w_{i}}{\sum_{i=1}^{N}\psi_{i}(x)}x(g - y_{0})
		\end{equation}
		where,
		\begin{equation}\label{psi}
		\psi_{i} = \exp(-{\frac{1}{2\sigma_{i}^{2}}(x - c_{i})^{2}})
		\end{equation}
		and,
		\begin{equation}\label{canonical}
		\tau \dot{x} = -\alpha_{x}x
		\end{equation}
	\end{frame}
	
	\begin{frame}{Working of DMP}
		\includegraphics[width=\textwidth]{images/dmp_no_f}
	\end{frame}
	\begin{frame}
		\includegraphics[width=\textwidth]{images/step_f}
	\end{frame}
	\begin{frame}
		\begin{figure}
			\includegraphics[scale=0.23]{images/f_x}
			\caption{Forcing term - X}
		\end{figure}

		\begin{figure}
			\includegraphics[scale=0.23]{images/f_y}
			\caption{Forcing term - Y}
		\end{figure}

	\end{frame}
	
	\begin{frame}{Analysis of the effects of the parameters used in DMP}
		\includegraphics[width=\textwidth]{images/n_bfs_}
	\end{frame}
	
	\begin{frame}
		\centering
		\includegraphics[scale=0.45]{images/dt_}
	\end{frame}
	
	\begin{frame}
		\includegraphics[width=\textwidth]{images/tau_}
	\end{frame}
	
	\begin{frame}{Inverse Kinematic Solver}
		
	\end{frame}
	
	\begin{frame}{Whole Body Motion Control}
		\begin{equation}
		m_{cap} = \frac{(\sigma_{min} - \sigma_{l})}{(\sigma_{h} - \sigma_{l})}
		\end{equation}

		\begin{equation}
		b_{cap} = \frac{(d - d_{l})}{(d_{h} - d_{l})}
		\end{equation}
		
		\begin{equation}
		v_{ee} = m_{cap}.v
		\end{equation} 
		
		\begin{equation}
		v_{b} = (1 - m_{cap}).v
		\end{equation} 

	\end{frame}
	
	\begin{frame}{Experimental Evaluations}
		
	\end{frame}
	
	\begin{frame}{Results}
	
	\end{frame}
	
	\begin{frame}{Conclusion}
	
	\end{frame}

\end{document}
